%% sample5.tex
%% Created by Laurence D. Finston (LDF) Wed 14 Jul 2021 03:19:08 PM CEST

%% * (1) Copyright and License.

%%%% This file is part of GNU 3DLDF, a package for three-dimensional drawing. 
%%%% Copyright (C) 2021 The Free Software Foundation, Inc.
   
%%%% GNU 3DLDF is free software; you can redistribute it and/or modify 
%%%% it under the terms of the GNU General Public License as published by 
%%%% the Free Software Foundation; either version 3 of the License, or 
%%%% (at your option) any later version. 

%%%% GNU 3DLDF is distributed in the hope that it will be useful, 
%%%% but WITHOUT ANY WARRANTY; without even the implied warranty of 
%%%% MERCHANTABILITY or FITNESS FOR A PARTICULAR PURPOSE.  See the 
%%%% GNU General Public License for more details. 

%%%% You should have received a copy of the GNU General Public License 
%%%% along with GNU 3DLDF; if not, write to the Free Software 
%%%% Foundation, Inc., 51 Franklin St, Fifth Floor, Boston, MA  02110-1301  USA

%%%% GNU 3DLDF is a GNU package.  
%%%% It is part of the GNU Project of the  
%%%% Free Software Foundation 
%%%% and is published under the GNU General Public License. 
%%%% See the website http://www.gnu.org 
%%%% for more information.   
%%%% GNU 3DLDF is available for downloading from 
%%%% http://www.gnu.org/software/3dldf/LDF.html. 

%%%% Please send bug reports to Laurence.Finston@gmx.de.
%%%% The mailing list help-3dldf@gnu.org is available for people to 
%%%% ask other users for help.  
%%%% The mailing list info-3dldf@gnu.org is for the maintainer of 
%%%% GNU 3DLDF to send announcements to users. 
%%%% To subscribe to these mailing lists, send an 
%%%% email with ``subscribe <email-address>'' as the subject.  

%%%% The author can be contacted at: 

%%%% Laurence D. Finston 
%%%% c/o Free Software Foundation, Inc. 
%%%% 51 Franklin St, Fifth Floor 
%%%% Boston, MA  02110-1301  
%%%% USA

%%%% Laurence.Finston@gmx.de



%% * (1)

\input eplain
\input epsf 
\nopagenumbers
\input colordvi

\enablehyperlinks[dvipdfm]
\hlopts{bwidth=0}

\font\small=cmr8
\font\smalltt=cmtt8
\font\medium=cmr10 scaled \magstephalf
\font\mediumbx=cmbx10 scaled \magstephalf
\font\mediumit=cmti10 scaled \magstephalf
\font\mediumtt=cmtt10 scaled \magstephalf
\font\bx=cmbx10
\font\large=cmr12
\font\largebx=cmbx12
\font\largeit=cmti12
\font\largett=cmtt12
\font\largesy=cmsy10 scaled 1200
\font\Largebx=cmbx12 scaled \magstephalf
\font\huge=cmr12 scaled \magstep2
\font\hugebx=cmbx12 scaled \magstep2
\font\mediumsy=cmsy10 scaled \magstephalf
\font\mediumcy=cmcyr10 scaled \magstephalf

\def\bigcirc{{\largesy\char"0E}}

\newdimen\twozerosdimen
\setbox0=\hbox{{\medium 00}}
\twozerosdimen=\wd0

\newcount\chapctr
\chapctr=0

\newcount\temppagecnt
\temppagecnt=0

\newcount\sectionctr
\sectionctr=0

\def\tocchapterentry#1#2#3{\line{\hbox to \twozerosdimen{\hss #2}. {\medium #1 \dotfill\ #3}}}
\def\tocsectionentry#1#2#3{\setbox0=\hbox{#2}\line{\hskip3em\ifdim\wd0>0pt{#2}. \fi{\medium #1} \dotfill\ #3}}

\def\Chapter#1#2{\global\advance\chapctr by 1\sectionctr=0%
\writenumberedtocentry{chapter}{\hlstart{}{bwidth=0}{#1}\Blue{#2}\hlend}{\the\chapctr}}
\def\Section#1#2{\global\advance\sectionctr by 1\edef\A{\the\chapctr .\the\sectionctr}%
\writenumberedtocentry{section}{\hlstart{}{bwidth=0}{#1}\Blue{#2}\hlend}{\A}}

%% * (1)

%% Uncomment for A4 portrait
\iffalse % \iftrue
\special{papersize=210mm, 297mm}
\hsize=210mm
\vsize=297mm
\fi

%% Uncomment for A4 landscape.
\iffalse % \iftrue
\special{papersize=297mm, 210mm}
\hsize=297mm
\vsize=210mm
\fi

%% Uncomment for DIN A3 portrait.
\iffalse % \iftrue
\special{papersize=297mm, 420mm} %% DIN A3 Portrait
\vsize=420mm
\hsize=297mm
\fi

%% Uncomment for A3 landscape.
\iftrue % \iffalse
\special{papersize=420mm, 297mm} %% DIN A3 Landscape
\vsize=297mm
\hsize=420mm
\fi

\advance\voffset by -1in
\advance\voffset by 2cm
\advance\hoffset by -1in
\advance\hoffset by 1.5cm
\advance\vsize by -3.5cm

\parindent=0pt

%% *** (3) 

\def\epsfsize#1#2{#1}

%% *** (3) 

\pageno=1
\footline={\hfil \folio\hfil}

\begingroup
\advance\hsize by -1in
\parskip=.5\baselineskip
\centerline{{\largebx Celestial Sphere Models 1}}
\vskip\baselineskip

\small
This document is part of GNU 3DLDF, a package for three-dimensional

Copyright (C) 2021 The Free Software Foundation

GNU 3DLDF is free software; you can redistribute it and/or modify 
it under the terms of the GNU General Public License as published by 
the Free Software Foundation; either version 3 of the License, or 
(at your option) any later version. 

GNU 3DLDF is distributed in the hope that it will be useful, 
but WITHOUT ANY WARRANTY; without even the implied warranty of 
MERCHANTABILITY or FITNESS FOR A PARTICULAR PURPOSE.  See the 
GNU General Public License for more details. 

You should have received a copy of the GNU General Public License 
along with GNU 3DLDF; if not, write to the Free Software 
Foundation, Inc.,\hfil\break
51 Franklin St, Fifth Floor, Boston, MA  02110-1301  USA

See the GNU Free Documentation License for the copying conditions 
that apply to this document.

You should have received a copy of the GNU Free Documentation License 
along with GNU 3DLDF; if not, write to the Free Software 
Foundation, Inc., 51 Franklin St, Fifth Floor, Boston, MA  02110-1301  USA

Please send bug reports to {\smalltt Laurence.Finston@gmx.de} 
The mailing list {\smalltt help-3dldf@gnu.org} is available for people to 
ask other users for help.  
The mailing list {\smalltt info-3dldf@gnu.org} is for sending 
announcements to users. To subscribe to these mailing lists, send an 
email with ``subscribe $\langle$email-address$\rangle$'' as the subject.  

The author can be contacted at:

Laurence D. Finston\hfil\break
c/o Free Software Foundation, Inc.\hfil\break
51 Franklin St, Fifth Floor \hfil\break
Boston, MA  02110-1301 USA\hfil\break
{\smalltt Laurence.Finston@gmx.de}
\vskip\baselineskip
\centerline{{\largebx Table of Contents}}
\advance\baselineskip by 2pt
%\advance\parskip by 2pt
\readtocfile
\vskip1.5\baselineskip

%% *** (3)

\medium
\Chapter{instructions}{Instructions}
\centerline{{\largebx Instructions}}
\hldest{fit}{}{instructions}
\advance\baselineskip by 4pt
\parskip=.5\baselineskip
{\bf PLEASE NOTE!}  The author has tried to ensure that the following
plans are correct, but as of August 30, 2021, he has not tested them
yet himself.  As mentioned above, this material is distributed {\bf without
a warranty}.  I recommend that users check it themselves before
investing a lot of time and effort into cutting out the paper model.

Any corrections will be gratefully received by the author.  Contact
information can be found on the title page.

The plan on pages represent a ``development'' of a sphere:  The individual figures
are ``flattened-out'' spherical biangles corresponding to $1/8$ of a sphere.

To use these plans, tape, tack or otherwise attach them to a sheet of
paper which should be robust, but not too thick.  I generally prefer tacking
to taping, where possible.

Fairly light, smooth
watercolor paper might be a good choice.  Bristol board or cardboard cannot be
used for this model, because the pieces need to be able to bend.  In addition,
if the paper is too thick, it may be difficult to attach the tabs.  For better
accuracy, it would be necessary to account for the thickness of the paper when
calculating the shape of the tabs.

The knife must be sharp as watercolor paper (or other heavy papers) will 
dull the blade quickly.
I have been using knives with disposable blades.  I've been meaning to
try sharpening them but haven't done so yet.  I therefore can't say
whether this will work.  It seems a shame to waste so many blades,
which is why I have a jar full of them.  They must be good for
something. 

It will be necessary to reattach the plans parts of them are cut out.

Make sure that the plan is attached smoothly or you will introduce
inaccuracies.  {\it Do not detach it or let it slip until you are
done!\/}  You will never get it back where it's supposed to go. 
However, with this model, this is only important for an individual piece,
since they aren't attached to each other.

If you use tape, please use the removable kind.  Ordinary masking tape will
damage the paper when it is removed.  Be aware that ``removable tape'' isn't
completely reliable, especially if left too long on the drawing.  Sometimes
it's possible to reuse pieces of it, which avoids wasting large amounts of
it. 
\vfil\eject
\endgroup


\begingroup
\advance\hoffset by 2cm
%% **** (4)

\Chapter{spherical biangles}{Spherical Biangles}
\centerline{{\largebx Spherical Biangles}}
\vskip\baselineskip
\hldest{fit}{}{spherical biangles}
\line{\epsffile{sample5.0}\hskip1cm\epsffile{sample5.1}\hskip1cm\epsffile{sample5.2}\hskip1cm
\epsffile{sample5.3}\hss}
\vfil\eject

%% **** (4)

\line{\epsffile{sample5.4}\hskip1cm\epsffile{sample5.5}\hskip1cm
\epsffile{sample5.6}\hskip1cm\epsffile{sample5.7}\hss}
\vfil\eject
\endgroup

%% ** (2) Panels

%% *** (3)

\Chapter{panels}{Panels (Isoceles Trapezoids and Triangles)}
\centerline{{\largebx Panels (Isoceles Trapezoids and Triangles)}}
\hldest{fit}{}{panels}
\vskip2\baselineskip
\line{\epsffile{sample5.12}\hss}
\vfil\eject

%% *** (3)

\line{\epsffile{sample5.13}\hss}
\vfil\eject

%% ** (2) Globes

\Chapter{globes}{Globes}
\centerline{{\largebx Globes}}
\hldest{fit}{}{globes}
\vskip2\baselineskip
\line{\epsffile{sample5.8}\hskip.25cm\epsffile{sample5.9}\hss}
\vfil\eject

\line{\epsffile{sample5.10}\hskip.25cm\epsffile{sample5.11}\hss}
\vfil\eject

%% ** (2) Testing

%% \line{\epsffile{sample5.14}\hss}
%% \vfil\eject

%% *** (3)

\begingroup
\advance\hsize by -.5\hsize
\advance\hsize by -6cm
\parskip=\baselineskip
\Chapter{copying}{Copying}
\hldest{fit}{}{copying}
%\advance\baselineskip by -6pt
\advance\hsize by 10in
\listing{gpl-3.0_a3.txt}
\endgroup

%% *** (3)

%% \line{\epsffile{sample5.14}\hss}
%% \vfil\eject


%% *** (3) End here

\bye

%% ** (2)

%% * (1) Local variables for Emacs.

%% Local Variables:
%% mode:TeX
%% eval:(outline-minor-mode t)
%% eval:(read-abbrev-file abbrev-file-name)
%% abbrev-mode:t
%% outline-regexp:"%% [*\f]+"
%% End:

